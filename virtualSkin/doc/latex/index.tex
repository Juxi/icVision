\hypertarget{index_intro}{}\section{Introduction}\label{index_intro}
VirtualSkin is a collection of tools intended to facilitate your research on complex robotic platforms such as humanoids. It is particularly well suited to reaching, grasping and manipulation, however some parts of the VirtualSkin distribution could certainly be applied to other research topics as well. The idea behind VirtualSkin is that having a cartesian space representation of the geometry of the robot's pose, which is light-\/weight and fast (provided by a forward kinematic model and attached geometries), is a useful thing to have. Such a representation, driven by streams of encoder positions from the real hardware (run online), can be used to protect the robot from damage resulting from collision with itself or with immovable objects in the environment. Moreover, it can be used offline to facilitate motion planning by searching for feasible motions much faster than real-\/time. Whether run online or offline, VirtualSkin reports collision information in bodypart-\/centric cartesian coordinates, which provides a pose-\/invariant representation of the location of collision events on the body.\hypertarget{index_contents}{}\section{What's In The Box}\label{index_contents}

\begin{DoxyItemize}
\item {\bfseries librobotModel.a} -\/ a static library for building and moving a kinematic robot model
\item {\bfseries libyarpFilter.a} -\/ a static library containing a port filter for YARP communication channels (contributed by Gregor Kaufmann -\/ also available as part of \char`\"{}idsiaYarpModules\char`\"{})
\item {\bfseries reflex} -\/ an executable that integrates these two libraries. It tracks a YARP robot (currently iCub or iCubSim but you can model any robot quickly and easily in xml), reports collision events (as described above). In case of collision, it takes control away from the user, executes a reflexive response to a previous collision free pose, then returns control to the user.
\item {\bfseries iCubBabble} -\/ a simple control program for the iCub robot (only) that issues random position or velocity commands.
\end{DoxyItemize}\hypertarget{index_upcoming}{}\section{Coming Soon}\label{index_upcoming}

\begin{DoxyItemize}
\item A new YARP RPC interface to move the robot model directly using position and velocity commands (facilitates planner-\/mode, decoupling the model from an external robot or simulated robot)
\item Collision events reported in a 2D parameterized \char`\"{}Skin Space\char`\"{} that preserves neighborhood information for adjacent body parts
\item Cartesian positions/orientations of arbitrary body parts published on the network via YARP
\end{DoxyItemize}\hypertarget{index_install}{}\section{Installation}\label{index_install}

\begin{DoxyEnumerate}
\item Make sure you have all dependencies installed and working.
\begin{DoxyItemize}
\item CMake $>$= 2.6 \href{http://www.cmake.org/}{\tt http://www.cmake.org/}
\item YARP $>$= 2.6 \href{http://eris.liralab.it/yarp/}{\tt http://eris.liralab.it/yarp/}
\item Qt $>$= 4 \href{http://qt.nokia.com/products/}{\tt http://qt.nokia.com/products/}
\item OpenGL (comes with Qt) \href{http://www.opengl.org/}{\tt http://www.opengl.org/}
\item FreeSOLID \href{http://www.win.tue.nl/~gino/solid/}{\tt http://www.win.tue.nl/$\sim$gino/solid/}
\end{DoxyItemize}
\item Clone the IDSIA IM-\/CLeVeR GIT repository using the following command: \begin{DoxyVerb}git clone http://git.ti-edu.ch/repogit/IM-CLeVeR.git PATH_TO_YOUR_LOCAL_GIT_REPOSITORY \end{DoxyVerb}
 \begin{DoxyNote}{Note}
The directory PATH\_\-TO\_\-YOUR\_\-LOCAL\_\-GIT\_\-REPOSITORY should not exist yet... it will be created. 

Questions concerning the GIT repository should be directed to -\/ \href{mailto:varun@idsia.ch}{\tt varun@idsia.ch}
\end{DoxyNote}

\item From PATH\_\-TO\_\-YOUR\_\-LOCAL\_\-GIT\_\-REPOSITORY/virtualSkin do the following:
\begin{DoxyItemize}
\item \begin{DoxyVerb}mkdir build \end{DoxyVerb}

\item \begin{DoxyVerb}cd build \end{DoxyVerb}

\item \begin{DoxyVerb}cmake .. \end{DoxyVerb}

\item \begin{DoxyVerb}(optional) ccmake .. \end{DoxyVerb}
 ...to tweak build options i.e. set build type to DEBUG, or to set paths to dependencies if cmake fails to find them
\item \begin{DoxyVerb}make \end{DoxyVerb}

\item \begin{DoxyVerb}(optional) make install \end{DoxyVerb}
 ...this will install build products to:
\begin{DoxyItemize}
\item PATH\_\-TO\_\-YOUR\_\-LOCAL\_\-GIT\_\-REPOSITORY/virtualSkin/bin
\item PATH\_\-TO\_\-YOUR\_\-LOCAL\_\-GIT\_\-REPOSITORY/virtualSkin/lib
\item PATH\_\-TO\_\-YOUR\_\-LOCAL\_\-GIT\_\-REPOSITORY/virtualSkin/include
\end{DoxyItemize}
\end{DoxyItemize}
\end{DoxyEnumerate}Once you have everything installed, you can try out the \hyperlink{reflex_demo}{Reflex Demonstration}

This project was developed during the fist year of my PhD at IDSIA's robotics lab.

2011-\/01-\/01, Kail Frank -\/ \href{mailto:kail@idsia.ch}{\tt kail@idsia.ch} 